\section*{Resumo}
% Parágrafo intro
Nos dias de hoje, o voluntariado é uma atividade que ganha cada vez mais destaque, pois promove não só o enriquecimento da sociedade como também a solidariedade e a valorização do meio-social. A participação neste tipo de ações permite ao voluntário o desenvolvimento de competências como a liderança e o trabalho em equipa, as quais são valorizadas no meio profissional. \par \smallskip
% Parágrafo problemática
A divulgação e inscrição nas mesmas é realizada normalmente através de redes sociais dedicadas a conexões sociais (por exemplo, Facebook, Twitter, Google+), que  não são vocacionadas para este tipo de ações, e em \textit{websites}, o que leva a uma descentralização da informação. \par \smallskip
% Parágrafo descrição do projeto
O presente projeto visa abordar este problema através do desenvolvimento de uma rede social orientada ao voluntariado, onde voluntários e organizações coabitam, dando-se foco à divulgação e inscrição neste tipo de ações mas mantendo os atributos típicos de uma rede social (interação entre utilizadores). O sistema é composto por uma REST API (Javascript) e duas aplicações cliente: uma aplicação \textit{mobile} (Android) orientada aos voluntários e uma aplicação \textit{web} (React) orientada às organizações. \par \smallskip
% Palavras chave
\textbf{Palavras-chave}: voluntariado, ações de voluntariado, \textit{web} API, aplicação \textit{mobile}, aplicação \textit{web}.