\section{\textit{Web App}}

No capítulo seguinte apresentam-se detalhes relativos à conceptualização e desenvolvimento da aplicação \textit{web}. Após introduzir o componente, serão descritos os seus requisitos, seguindo com a descrição da arquitetura e módulos constituintes.

\par \medskip

A aplicação \textit{web} é responsável por estabelecer uma interface sobre a qual as organizações podem interagir com a plataforma, disponibilizando às mesmas ferramentas que possibilitam a realização de operações como por exemplo a criação de \textit{posts} ou a realização de pesquisas sobre a plataforma.

\par \medskip

Este componente foi desenvolvido usando a tecnologia React em conjunto com algumas bibliotecas adicionais, disponibilizadas sobre a plataforma NPM.

\subsection{Tecnologia}

Face à proposta de projeto, ouve uma alteração relativamente à tecnologia usada para desenvolver este componente. Numa abordagem inicial, foi tomada a decisão de desenvolver este módulo usando a ferramenta Angular.js. \par \medskip

No entanto, após nova avaliação, optou-se por utilizar a tecnologia React. Esta alteração foi efetuada após verificar que React é a tecnologia mais favorecida no mercado à data da realização do projeto. \par \medskip

\subsection{Funcionalidades e requisitos}

Tendo em conta que este módulo é utilizado diretamente por um cliente humano, foi necessário desenvolver o mesmo de maneira a que este fosse o mais intuitivo e agradável de utilizar possível. 

\par \medskip

Para além deste requisito não funcional, é também necessário que este módulo cumpra com as seguintes necessidades:

\begin{itemize}
	\item possibilitar a realização de operações simples de pesquisa (de voluntários, organizações, entre outros) sobre a API e apresentar os resultados destas. Estas pesquisas não necessitam de autenticação por parte do utilizador e estão disponíveis a todos os clientes da plataforma;
	\item apresentação de uma página que indica a clientes da plataforma a aplicação sobre a qual os mesmos se podem registar e autenticar (remetendo voluntários para a aplicação móvel e organizações para a página de \textit{login} deste módulo);
	\item permitir a autenticação de organizações e fornecer operações às mesmas que possibilitam a interação entre estas e a plataforma (criação de eventos e \textit{posts}, seguimentos de outros utilizadores, etc.).
\end{itemize}

\bigskip \bigskip

Este componente encontra-se em desenvolvimento na data da entrega da versão beta. Como tal, este capítulo apenas irá ser concluído no momento da entrega do projeto final.