\pagebreak
\hspace{0pt}
\vfill
{
	\section{\textit{Web App}}
	
	Neste capítulo é abordado o desenvolvimento da aplicação \textit{web}. É realizada uma introdução, apresentando os objetivos e funcionalidades da mesma. São mostrados detalhes relativos à utilização da aplicação e, de seguida, apresenta-se o modelo de arquitetura utilizado no desenvolvimento da mesma. Por fim, refere-se como aceder à aplicação e define-se o seu processo de \textit{deployment}.
	
	\par \smallskip
	
	A aplicação \textit{web} é responsável por estabelecer uma interface sobre a qual as organizações podem interagir com a plataforma, disponibilizando às mesmas ferramentas que possibilitam a realização de operações como por exemplo a criação de \textit{posts} ou a realização de pesquisas sobre a plataforma.
	
	\par \smallskip
	
	Foram definidos alguns requisitos chave no início da conceptualização da aplicação, como por exemplo:
	
	\begin{itemize}
		\item disponibilizar meios para consultar os \textit{posts}, eventos e outros utilizadores da plataforma e interagir com os mesmos; 
		\item permitir às organizações editarem o seu perfil;
		\item possibilitar que as mesmas possam criar e editar \textit{posts} e eventos;
		\item apresentar contactos de voluntários interessados em eventos pertencentes à organização autenticada.
	\end{itemize}
	
	Numa fase inicial do projeto, foi considerada a opção de desenvolver a aplicação \textit{web} usando a \textit{framework} Angular.js. Contudo, após nova avaliação, optou-se por utilizar a biblioteca React. 
	
	\medskip
	
	Esta decisão foi principalmente influenciada devido ao facto de que, tal como já foi referido, React é a tecnologia mais utilizada para desenvolver \textit{front-end} à data da realização do projeto. Outra fator determinante nesta decisão foi a experiência dos autores com esta ferramenta.
}
\vfill
\hspace{0pt}
\pagebreak

\subsection{Utilização da \textit{Web App}}

Levando em consideração que este componente foi desenvolvido especificamente para organizações, a maioria das operações implicam que um utilizador organização esteja autenticado (através da interface demonstrada na Figura 9).

\begin{figure}[h]
	\centering
	\includegraphics[scale=.38]{web_app_login_page}
	\caption{Interface de autenticação.}
\end{figure}

Nesta versão da aplicação, é permitido que sejam registados utilizadores do tipo organização. Contudo, numa versão publicada da plataforma, o registo deste tipo de utilizadores seria realizado através do contacto direto dos mesmos com os gestores da aplicação de maneira a que apenas organizações fidedignas pudessem ter um perfil na plataforma.

\par \medskip

\begin{figure}[h]
	\centering
	\includegraphics[scale=.39]{web_app_dashboard}
	\caption{Painel principal da aplicação.}
\end{figure}

Após autenticação na aplicação é disponibilizado um painel principal (consultar figura 10) onde é possível navegar entre as várias páginas da aplicação e realizar as operações pretendidas.

\par \medskip

\subsection{Arquitetura}

A arquitetura da aplicação é composta principalmente por dois módulos: API e Componentes e ainda a classe principal da aplicação~\cite{Stefanov2016}.

\par \medskip 

Similar ao funcionamento do módulo com o mesmo nome na aplicação \textit{mobile}, o módulo API disponibiliza operações que realizam pedidos HTTPS à \textit{web} API. O módulo Componentes contém a implementação de todos os componentes React apresentados ao utilizador, desde as páginas em si a componentes que são utilizados nestas. 

\par \medskip

A classe principal da aplicação é responsável não só por instanciar os serviços da API mas também por definir o roteamento da aplicação \textit{web}.

\subsubsection{API}

O módulo API é composto por um conjunto de serviços que disponibilizam operações que necessitam de realizar pedidos HTTPS à API (como por exemplo a solicitação de \textit{posts} ou a criação de um evento).

\par \medskip

Para cada entidade (voluntários, organizações, \textit{posts} e eventos) existe um serviço onde são definidas as operações possíveis de efetuar sobre a mesma. Todos os serviços utilizam uma classe auxiliar que contém a implementação de como efetuar pedidos HTTPS consoante o seu método (neste caso, GET, PUT, POST e DELETE).

\subsubsection{Componentes}

Tal como já referido, o módulo Componentes é responsável por tratar a estruturação e apresentação da interface da aplicação assim como lidar com operações de entrada de dados por parte do utilizador. Como tal, este contém a definição de:

\begin{itemize}
	\item \textbf{componentes página}. Estes componentes definem os sub-componentes que constituem a página (por exemplo, na figura 11, a página dos eventos é constituída por um formulário para criar um novo evento e a lista dos eventos existentes na plataforma);
	\item \textbf{componentes específicos} por página, como por exemplo, uma lista de \textit{posts} ou um formulário para criar eventos;
	\item \textbf{componentes utilitários}, responsáveis por apresentar certos aspetos comuns da aplicação.
\end{itemize}

\begin{figure}[h]
	\centering
	\includegraphics[scale=.52]{components_explained}
	\caption{Exemplo de tipos de componentes na página dos eventos.}
\end{figure}

Os componentes deste projeto que necessitam de atualizar o seu estado após um determinado espaço de tempo (por exemplo, consultar a API de maneira a verificar a existência de novos \textit{posts}) foram desenvolvidos através da definição de classes que extendem de React Component~\cite{ReactJS2020} e que re-implementam os métodos necessários para o seu funcionamento (por exemplo, \textit{render} e \textit{componentDidMount}).

\medskip

Todos os outros componentes foram desenvolvidos através do uso de React Hooks~\cite{ReactJS2019} (como o \textit{useState} e o \textit{useEffect}), que simplificam o processo de implementação através da possibilidade do acesso direto a funções que alteram o valor dos estados do componente.

\subsubsection{Classe Principal}

A classe principal da aplicação instancia os serviços usados pelos componentes para efetuar pedidos à API e define também o roteamento da aplicação \textit{web}. No modo não autenticado, o acesso a quase todas as rotas da aplicação \textit{web} é restringido. Apenas quando o cliente realiza autenticação com sucesso é permitido ao mesmo aceder às funcionalidades principais da aplicação.

\subsection{Implantação da aplicação}

A aplicação foi implantada utilizando uma máquina virtual do serviço \textit{Compute Engine} da \textit{Google Cloud Platform}. Foram instaladas as ferramentas necessárias para executar a aplicação nesta máquina (nomeadamente \textit{Node.js}) e a mesma encontra-se instanciada numa das portas da VM. 

\par \medskip

Tendo em consideração que a API encontra-se implantada noutra máquina, é necessário lidar com a realização de pedidos CORS (Cross-Origin Resource Sharing), isto é, pedidos efetuados pela aplicação \textit{web} a um domínio externo (neste caso, ao domínio atribuído à API)~\cite{MDN2019}. 

\par \medskip

De maneira a lidar com estes, na máquina onde é executada a aplicação está instanciado um servidor \textit{web} Nginx. Este está configurado de maneira a redirecionar todos os pedidos que comecem com o preâmbulo /api para a máquina onde está implantada a \textit{web} API. Todos os outros pedidos são redirecionados para a porta onde está a ser executada a aplicação \textit{web}. 

\par \medskip

Como consequência deste redirecionamento, todos os pedidos efetuados pela aplicação são realizados para o seu próprio domínio (lidando com as restrições impostas pelo CORS) e redirecionados pelo servidor \textit{web} para a API.

\par \medskip

De maneira a associar um nome de domínio ao endereço IP da máquina onde está a ser executado o servidor \textit{web}, foi utilizada a plataforma Duck DNS, que permite reservar sem qualquer custo um número limitado de \textit{domain names}.

\par \medskip

Por fim, e utilizando as ferramentas disponibilizadas pela plataforma CertBot, foi automaticamente emitido um certificado SSL e alterada a configuração do servidor \textit{Nginx} de maneira a que a plataforma aceitasse apenas comunicações através do protocolo HTTPS, garantindo segurança ponto-a-ponto entre máquina cliente e servidor \textit{web}.

\par \medskip

A aplicação encontra-se acessível em \url{https://tribute-app.duckdns.org/}.

\begin{figure}[h]
	\centering
	\includegraphics[scale=.60]{web_app_deployment}
	\caption{Implantação da aplicação \textit{web}.}
\end{figure}
