\subsection{Tecnologias e ferramentas}
As seguintes tecnologias foram utilizadas durante o desenvolvimento do projeto:
\begin{itemize}
	\item \textbf{Javascript}: Principal linguagem para programação \textit{client-side} em \textit{browsers}; Esta é tipicamente utilizada em conjunto com ferramentas como o HTML e CSS para implementar a funcionalidade de uma página \textit{web};
	\item \textbf{Node.js}: Interpretador e ambiente de execução para Javascript normalmente utilizado para executar código sem ser num cliente \textit{browser};
	\item \textbf{NPM}: \textit{package manager} do Javascript/Node.js;
	\item \textbf{Typescript}: linguagem \textit{open source} que extende de Javascript, adicionando um sistema de tipos e outros recursos. Esta linguagem compila para Javascript;
	\item \textbf{Express}: \textit{web framework} para Node.js. Auxilia o processo de \textit{routing} e definição de \textit{endpoints}, encapsulando aspetos do HTTP, para tornar mais fácil o desenvolvimento de \textit{Web} APIs;
	\item \textbf{Passport}: \textit{Middleware} de autenticação usado em conjunto com o Express para simplificar o processo de autenticação e gestão de sessões de utilizadores;
	\item \textbf{MongoDB}: Base de dados noSQL baseada em documentos JSON. Tipicamente integrada com Javascript devido à natureza dos seus documentos;
	\item \textbf{Android}: SO \textit{open source} para dispositivos móveis desenvolvido pela Google. Atualmente, cerca de 75\% dos dispositivos móveis usam este SO;
	\item \textbf{Kotlin}: Linguagem de programação desenvolvida pela JetBrains que compila para a JVM (\textit{Java Virtual Machine}). Atualmente, o Kotlin é uma das linguagems oficiais para desenvolvimento de aplicações Android;
	\item \textbf{Android Jetpack}: Conjunto de ferramentas e bibliotecas que auxiliam a implementação e desenvolvimento de software para o sistema operativo móvel Android;
	\item \textbf{Volley}: Biblioteca \textit{open-source} desenvolvida para simplificar a realização de pedidos HTTP no ambiente Android;
	\item \textbf{Glide}: \textit{Biblioteca} utilizada para efetuar o carregamento de imagens em aplicações Android;
	\item \textbf{React}: Biblioteca \textit{open-source} de Javascript usada para desenvolver aplicações \textit{web};
	\item \textbf{Bootstrap}: Biblioteca \textit{open-source} que auxilia o desenvolvimento de interfaces de utilizador, tipicamente utilizada em ambientes \textit{web};
	\item \textbf{Nginx}: Servidor \textit{web} utilizado para hospedar plataformas;
	\item \textbf{Duck DNS}: Serviço dinâmico de DNS, que permite atribuir um nome de domínio à plataforma gratuitamente;
	\item \textbf{CertBot}: Cliente open source usado para obter certificados SSL fornecidos gratuitamente pela autoridade \textit{Let's Encrypt}.
\end{itemize}