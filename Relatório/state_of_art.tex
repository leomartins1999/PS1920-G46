\section{Formulação do problema}

Este capítulo começa por incidir sobre as dificuldades encontradas na divulgação e inscrição por parte de voluntários em ações de voluntariado. São levantados alguns requisitos para o projeto e apresentadas algumas soluções possíveis, sendo as suas vantagens e desvantagens ponderadas para chegar à solução final. Por fim, são elaborados alguns requisitos funcionais do projeto face à solução definida, bem como a justificação da definição da arquitetura do mesmo. \par \medskip

Tal como já foi referido, a interação voluntário-organização é tipicamente feita através de dois tipos de plataformas: redes sociais e \textit{websites} das organizações.  \medskip

As redes sociais, por não serem, por desenho, vocacionadas para este tipo de ações, apresentam alguns problemas de utilização, como filtragem de informação e integração de múltiplas plataformas de voluntariado na mesma rede social.  \medskip

Por norma, cada organização tem o seu próprio \textit{website}, algo que complica o processo de navegação do voluntário, caso este esteja interessado em colaborar com múltiplas associações de voluntariado. 

\subsection{Requisitos não funcionais}
Tendo em conta os problemas apresentados, o nosso projeto tem como objetivo desenvolver uma plataforma onde é disponibilizada informação relativa a ações de voluntariado, desde eventos existentes a perfis de organizações. Esta deve também auxiliar o processo de candidatura/inscrição em ações deste tipo e promover a interação entre utilizadores. \bigskip

A seguir descrevem-se duas plataformas que possuem objetivos semelhantes aos do presente projeto.

\subsection{Estado da Arte}

A Bolsa de Voluntariado~\cite{bolsa_voluntariado} é um projeto lançado em 2006 pela associação ENTRAJUDA com o objetivo de facilitar a procura de trabalho voluntário. \medskip

Este objetivo é concretizado através duma plataforma \textit{web} que serve de ponto de encontro entre a procura e oferta de oportunidades de voluntariado. A plataforma permite consultar ações que irão decorrer, oferecendo ainda a possibilidade aos utilizadores de as filtrarem consoante os seus interesses e visa também facilitar o processo de candidatura às mesmas. \medskip

A plataforma \textit{Online Volunteering}~\cite{online_volunteering}, desenvolvida pela UN (\textit{United Nations}) e lançada em 2000, é uma plataforma que, através do voluntariado \textit{online}, pretende reunir voluntários de múltiplas origens de maneira a auxiliarem na resolução de desafios tecnológicos das mais variadas áreas. \medskip

Esta aplicação permite a filtragem das oportunidades consoante a área de interesse e também auxilia o processo de candidatura às mesmas. 

\subsection{Conclusão}

Apesar destas plataformas disponibilizarem informação relativa a oportunidades de realizar trabalho voluntário e das mesmas auxiliarem a inscrição nestas ações, estas não promovem a interação entre utilizadores, que é fulcral para o crescimento de uma comunidade solidária, que leva ao aumento do número de participantes em ações de voluntariado. \medskip

Uma solução aplicável poderia ser o desenvolvimento de uma ferramenta que aplicasse técnicas de \textit{web scraping}. Contudo, o desenvolvimento de uma ferramenta desta natureza depende da existência de fontes de informação externas e também não cumpre com a necessidade de haver interação entre utilizadores. \medskip

Uma rede social com foco em ações de voluntariado cumpre todos os requisitos descritos. Contudo, é necessário haver a migração de utilizadores para esta plataforma, algo que não é possível garantir. \medskip

Dadas as soluções apresentadas (e as suas vantagens/desvantagens), foi tomada a decisão de desenvolver uma rede social de voluntariado.

\subsection{Requisitos Funcionais}
Tendo em conta o desenvolvimento de uma rede social de voluntariado, foram elaborados alguns requisitos funcionais do projeto:

\begin{itemize}
	\item permitir a voluntários registarem-se, criarem um perfil e interagir com a plataforma (através da criação de \textit{posts} e sinalização de interesse em eventos);
	\item possibilitar às organizações solicitarem o registo na plataforma e também permitir às mesmas realizarem \textit{posts} e criarem eventos;
	\item mostrar às organizações os voluntários interessados nos seus eventos e disponibilizar um contacto dos mesmos (por exemplo, e-mail);
	\item garantir aos utilizadores da plataforma o acesso a interações como o seguimento de utilizadores, gostarem \textit{posts}, entre outros;
\end{itemize}

Levando em consideração os requisitos funcionais enumerados, foi tomada a decisão de elaborar o projeto em três componentes: 
\begin{enumerate}
	\item uma Web API, responsável por implementar as funcionalidades necessárias para as aplicações cliente terem o comportamento pretendido. Este módulo foi desenvolvido face à necessidade de expôr os dados de forma comum a todas as aplicações cliente.
	\item Uma aplicação cliente móvel (\textit{Android}) orientada aos voluntários, onde os mesmos podem interagir com a plataforma.	A decisão de desenvolver uma aplicação móvel para este sistema operativo foi tomada devido ao mesmo ser o mais comum no mercado móvel atualmente.
	\item Uma aplicação \textit{web} desenvolvida para as organizações interagirem com a plataforma, sendo que esta opção foi tomada para que múltiplos utilizadores possam gerir mais facilmente o perfil de uma organização.
\end{enumerate}