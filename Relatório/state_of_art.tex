\section{Estado da arte}
Tal como já foi referido, a interação voluntário-organização é tipicamente feita através de dois tipos de plataformas: redes sociais e \textit{websites} das organizações. \par \bigskip
As redes sociais, por não serem, por desenho, vocacionadas para este tipo de ações, apresentam alguns problemas de utilização, como filtragem de informação e integração de múltiplas plataformas de voluntariado na mesma rede social. \par \bigskip
Por norma, cada organização tem o seu próprio \textit{website}, algo que complica o processo de navegação do voluntário, caso este esteja interessado em múltiplas associações. \par \bigskip
A seguir descrevem-se duas plataformas que possuem objetivos semelhantes aos do presente projeto: \par \bigskip

A Bolsa de Voluntariado$^{[3]}$ é um projeto lançado em 2006 pela ENTRAJUDA com o objetivo de facilitar a procura de trabalho voluntário.  \par \medskip
Este objetivo é concretizado através duma plataforma \textit{web} que serve de ponto de encontro entre a procura e oferta de oportunidades de voluntariado. A plataforma permite consultar ações que irão decorrer, oferencendo ainda a possibilidade aos utilizadores de as filtrarem consoante os seus interesses e visa também facilitar o processo de candidatura às mesmas. \par \bigskip

A plataforma \textit{Online Volunteering}$^{[4]}$, desenvolvida pela UN (\textit{United Nations}) e lançada em 2000, é uma plataforma que, através do voluntariado \textit{online}, pretende reunir voluntários de múltiplas origens de maneira a auxiliarem na resolução de desafios tecnológicos das mais variadas áreas.  \par \medskip
Esta aplicação permite a filtragem das oportunidades consoante a área de interesse e também auxilia o processo de candidatura às mesmas. \bigskip

\subsection{Análise}
A principal problemática presente nestas plataformas é o facto de as mesmas realizarem o que é uma divulgação passiva (apresentar ações solicitadas pelo utilizador) em vez daquilo que é uma divulgação ativa (sugerir aos utilizadores ações de possível interesse). \par \bigskip
Esta pode ser combatida através do uso de mecanismos de interação similares aos das redes sociais, como ferramentas de “seguimento” de organizações ou tipos de ações. Essas ferramentas irão simplificar o processo de executar a divulgação ativa, e como tal, a personalização da experiência do uso da aplicação de utilizador para utilizador.

\subsection{Requisitos funcionais}

(todo)