\section{Conclusão}

O presente projeto trata o desenvolvimento de uma rede social de voluntariado, sendo que esta possibilita aos seus utilizadores criarem um perfil e interagirem com outros através de \textit{posts} e seguimentos. Contudo, o ênfase deste projeto é divulgar este tipo de ações, e como tal, permite a entidades organizadoras de eventos marcarem ações e obterem contatos de voluntários interessados em participar nos mesmos.

\par \medskip

Tendo isto em consideração, foram desenvolvidos três componentes:

\begin{itemize}
	\item uma \textit{web} API, que serve como fonte de dados para as aplicações cliente e que trata de definir e implementar as operações da plataforma;
	\item uma aplicação cliente móvel, que permite a voluntários consultarem a plataforma e registarem-se, criarem um perfil e interagirem com outros utilizadores;
	\item uma aplicação cliente \textit{web} que permite a organizações criarem o seu perfil e marcarem ações de voluntariado.
\end{itemize}

Contudo, este projeto teve alguns obstáculos, nomeadamente a seleção e utilização de um \textit{stack} tecnológico e o desenho e implementação da arquitetura da plataforma.

\par \medskip

Estes obstáculos apresentam-se porque ao longo da licenciatura é típico a imposição de um \textit{stack} tecnológico e a definição da arquitetura do sistema a desenvolver. Dada esta alteração de contexto, ao longo do projeto houve uma necessidade acrescentada de consultar não só o orientador como também artigos \textit{web}, livros e outros recursos de maneira a utilizar as ferramentas corretamente.

\par \medskip

Dado que o objetivo do projeto é desenvolver uma plataforma que poderia ser utilizada num contexto real, durante o desenvolvimento das interfaces de utilizador procurou-se dar a mesma importância ao aspeto funcional e visual das mesmas de maneira a que estas fossem simples, intuitivas e apelativas visualmente, um obstáculo novo para os autores. De maneira a ultrapassar este foram consultados guias de desenvolvimento de interfaces de utilizador.

\par \medskip

Apesar dos obstáculos apresentados, o projeto dá-se como concluído e este cumpre todos os requisitos funcionais definidos inicialmente, sendo que esta plataforma, com ligeiras alterações, poderia ser aplicada num contexto real e não apenas no contexto da elaboração de um trabalho final de curso. 

\newpage

\subsection{Trabalho futuro}

Contudo, e dada a natureza das redes sociais, estas encontram-se em constante desenvolvimento, e vão existir sempre melhorias a realizar e funcionalidades novas a desenvolver.

\par \medskip

De seguida, enumeram-se algumas destas:

\begin{itemize}
	\item aprimoramento do código desenvolvido nas aplicações cliente;
	\item desenvolvimento de testes unitários;
	\item adição de novas funcionalidades à plataforma - como comentários em \textit{posts}, noção de voluntário participante em evento e \textit{curriculum vitae} de voluntariado, que recolhe eventos em que voluntário participou;
	\item integração de registo e autenticação usando contas existentes noutras plataformas (através de OAuth 2.0~\cite{IETF2020}), nomeadamente, Facebook e Google; 
	\item melhoramento das soluções de \textit{deployment} aplicadas na plataforma (através do uso de técnicas de balanceamento de carga e de escalonamento automático).
\end{itemize}