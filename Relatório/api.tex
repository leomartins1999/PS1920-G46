\section{API}

(escrever introduçao dps xd)

\medskip \par

A \textit{web} API é responsável por estabelecer um serviço RESTful com o qual é possível comunicar sobre HTTPS e que implementa as funcionalidades da plataforma. Esta constitui o \textit{back-end} do projeto e como tal, funciona como fonte de dados para as aplicações cliente.

\medskip \par

Este módulo necessita de cumprir um conjunto de requisitos:

\begin{itemize}
	\item implementar as funcionalidades pretendidas da plataforma, como obtenção de listas de voluntários, registo por parte de utilizadores e outras operações;
	\item interagir com a base de dados;
	\item comunicar sobre HTTPS e suportar autenticação por parte dos clientes;
\end{itemize}

\medskip \par

Tendo em conta o \textit{stack} tecnológico selecionado para desenvolver este projeto, a API foi  desenvolvida em Typescript (sendo que o código é compilado para Javascript) e a mesma é instanciada usando Node.js.

\subsection{Funcionalidades}

(operações (n esquecer hosting de imagens))

(referir que definição de endpoint está na wiki)

(autenticação)

(paginação)

\subsection{Arquitetura}

(geral arquitetura)

\begin{figure}[h]
	\centering
	\includegraphics[scale=.35]{api_architeture}
	\caption{Diagrama de arquitetura da API}
\end{figure}

(explicamos diagrama)

(explicar classe main da aplicação - passagem router, etc.)

\subsubsection{Controladores}

(explicar funcionalidade)

(acesso a serviço correspondente)

(passagem dos parametros no body/query/... para a função de serviço... etc)

\subsubsection{Serviços}

(funcionalidade)

(relação de hierarquia com base service)

(chamada a repositórios)

\subsubsection{Repositórios}

(funcionalidade)

(relação com base repository e explicar base repository)

(falar de base repository e mongoquery)

\subsubsection{Autenticação}

(uso de passport.js)

(suporta autenticação à base de cookies)

\subsection{Base de dados}

(acesso à base de dados através da driver)

(talk about mongodb)

\subsection{Deployment da API}

xd

\iffalse

A \textit{Web} API (Figura 5) é composta por dois módulos principais (API e Service) e um conjunto de módulos auxiliares. Enquanto que a API lida com pedidos e respostas HTTP, o \textit{Service} é responsável por implementar a lógica do sistema. Existem ainda os \textit{repositórios} (sendo que cada um se responsabiliza por implementar a sua própria lógica de acesso à base de dados) e o \textit{índice de imagens} (responsável por tratar todas as operações associadas às mesmas).~\cite{nodejs_the_right_way}\par \medskip

\begin{figure}[h]
	\centering
	\includegraphics[scale=.35]{api_architeture}
	\caption{Diagrama de arquitetura da API}
\end{figure}

\subsection{API}
Este módulo é responsável por definir os \textit{endpoints} e lidar com a receção e envio dos pedidos/respostas HTTP. Cada \textit{endpoint} tem associada uma função que efetua a chamada ao módulo de serviço correspondente à operação solicitada pelo utilizador. \par \medskip

Encontra-se anexado a este relatório a lista de endpoints da API (anexo 1). Foram implementados os \textit{endpoints} que os autores consideraram necessários para garantir o funcionamento da plataforma. É de notar que todos os pedidos definidos têm o preâmbulo \textit{/api} e que os que começam por \textit{/auth} necessitam de autenticação prévia por parte do cliente da API. \par \medskip

Para além das funções já referidas, são também definidos (e utilizados) \textit{middlewares} nesta camada de maneira a garantir o cumprimento da necessidade de autenticação aquando de certas operações. A API é ainda responsável por transformar os parâmetros dados pelo cliente através do \textit{URL}, da \textit{query string} e do corpo do pedido (e do objecto sessão quando aplicável) para objetos conhecidos pelo \textit{serviço}. \medskip

\subsection{Serviço}
O módulo \textit{Service} cumpre a função de implementar a lógica da aplicação, isto é, garantir que as operações chamadas a partir da API se materializam em mudanças no modelo deste componente, seja a nível de base de dados ou tratamento de operações relativamente às imagens, entre outros. \par \medskip

Este módulo tem as seguintes responsabilidades:
\begin{itemize}
	\item verificar a validade dos parâmetros fornecidos pela API;
	\item executar as chamadas necessárias aos módulos adjacentes (\textit{repositories} e \textit{pictures}) de maneira a cumprir a operação solicitada (quer esta seja a inserção em base de dados ou a recolha de informações para verificar a lógica necessária);
	\item implementar a lógica da aplicação, isto é, quando necessário, efetuar as verificações necessárias e gerar os objetos a colocar/alterar na base de dados.
\end{itemize}

\subsection{Repositórios e acesso a base de dados}
Para cada índice na base de dados, existe uma variação de implementação de \textit{Repository}. Estas implementações fornecem métodos específicos para inserir/modificar/apagar entradas dos índices (coleção de documentos, como por exemplo: voluntários ou organizações) aos quais se referem. Estes repositórios, por sua vez, têm acesso a uma instância de \textit{BaseRepository} gerada especificamente para si (isto é, inicializada com argumentos personalizados consoante o índice). \par \medskip

\textit{BaseRepository} é uma infra-estrutura implementada de maneira a ser genérica para todos os índices e que fornece implementações das operações CRUD usuais como também a possibilidade de gerar os índices de pesquisa necessários para efetuar pesquisas por valor de campos nas tabelas. \par \medskip

\subsection{Base de Dados}
O motor de base de dados escolhido foi o MongoDB que é um motor \textit{noSQL}. Este apresenta um modelo não relacional, tipicamente utilizado no desenvolvimento de redes sociais. Esta escolha é justificada pelas seguintes razões:

\begin{itemize}
	\item flexibilidade. Tendo em conta que as plataformas desta natureza têm tipicamente um crescimento rápido, é necessário que a implementação de novas funcionalidades seja um processo ágil. Num modelo \textit{SQL}, por vezes é despendido tempo de implementação a efetuar estas alterações, algo que não é necessário neste caso porque os documentos não têm um esquema fortemente definido, podendo ser inseridos dinamicamente (dois documentos com características diferentes podem ser inseridos no mesmo índice);
	\item integração com Javascript. Os documentos inseridos nesta base de dados têm um formato JSON, algo que se integra facilmente com esta tecnologia.
\end{itemize} 

\subsection{Autenticação}
A componente autenticação deste módulo foi desenvolvida através da ferramenta Passport.js. Esta é tipicamente utilizada em conjunto com Express e permite a fácil implementação dum objeto \textit{sessão} através do uso de \textit{cookies}. \par \medskip

Aquando do pedido de autenticação bem sucedido por parte de um cliente da API, são guardadas um conjunto de características identificadoras da sessão do mesmo para que se possa personalizar a sua experiência, e neste caso, permitir ao mesmo que possa fazer o carregamento de imagens, criar um \textit{post}, entre outros. \par \medskip

\subsection{Imagens}
De maneira a implementar na sua totalidade um conceito de rede social, nesta aplicação é permitido o carregamento de imagens para a plataforma. Estas são divididas por índices, consoante a sua classe (aquilo que elas referem) e guardadas. \par \medskip

O carregamento de imagens para esta plataforma é apenas possível para utilizadores autenticados, no entanto, o descarregamento das mesmas é possível a qualquer cliente da API.

\subsection{Paginação e limitação de resultados}
Todas as operações da API que tenham como resultado um conjunto de entidades (por exemplo, um conjunto de \textit{posts}), retornam por omissão um número fixo de resultados. Contudo, a nossa API suporta paginação e \textit{skipping} (isto é, saltar resultados) através dos parâmetros \textit{limit} e \textit{skip} da \textit{query string}.

\subsection{Dados e imagens de utilizadores}
Dada a natureza deste projeto, todos os dados (perfis/fotos/\textit{posts}) serão públicos para todos, e como tal, aquando do registo nas aplicações cliente, os utilizadores das mesmas serão sensibilizados relativamente à visibilidade dos seus dados.

\subsection{Documentação e definição da API}
Toda a documentação sobre as regras e indicações relativamente à utilização desta API encontram-se na \textit{wiki} do repositório do projeto, sendo que o acesso será avaliado quando solicitado aos autores. Esta \textit{wiki} contém informações relativamente à definição mais explícita dos \textit{endpoints}, erros, entre outros.

\fi