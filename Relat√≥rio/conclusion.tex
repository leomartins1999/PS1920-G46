\section{Conclusão}
Este projeto trata o desenvolvimento uma rede social de voluntariado, sendo que para tal, foram implementados três módulos: uma \textit{web} API, uma aplicação móvel e uma aplicação \textit{web}. Para cada um destes módulos foi também necessário definir a sua arquitetura e, quando aplicável, o seu design gráfico.

\medskip

Foram apontadas algumas conclusões durante a realização do componente API:
\begin{itemize}
	\item apesar dos conhecimentos relativamente à implementação de uma API já obtidos ao longo da licenciatura, tivemos alguma dificuldade no desenho e definição de requisitos funcionais da API, dado o maior nível de complexidade comparado com projetos de disciplinas anteriores. Como tal, e ao longo do projeto, têm sido realizadas iterações à mesma de maneira que esta seja o mais adequada possível;
	\item a API, apesar de ser auto suficiente e não necessitar de outros módulos para funcionar, a mesma foi desenvolvida para comunicar em primeira mão com os outros módulos deste projeto. Como tal, é expectável que seja necessário definir \textit{endpoints} adicionais consoante as necessidades das aplicações cliente, como por exemplo a definição de um \textit{endpoint} que devolve os posts realizados por utilizadores que um certo utilizador autenticado segue.
\end{itemize}

\medskip

Relativamente ao desenvolvimento da aplicação \textit{mobile}, apresentam-se também algumas inferências.

\begin{itemize}
	\item Ao longo do lecionamento de unidades curriculares que entram em contacto com o desenvolvimento deste tipo de aplicações (nomeadamente, Programação em Dispositivos Móveis) diretamente, é sempre dado um ênfase não ao aspeto e sim à funcionalidade da mesma. Contudo, e dada a natureza deste projeto, foi dado um cuidado acrescentado à interface de utilizador.
\end{itemize}

\medskip

A utilização de \textit{stacks} tecnológicos não impostos por unidades curriculares da licenciatura e sim selecionados pelos autores do projeto levou por vezes à necessidade dos autores consultarem não só o orientador do projeto como também artigos \textit{web}, livros e outros recursos de maneira a tentar ultrapassar obstáculos encontrados na implementação do projeto.

\medskip

Outra problemática encontrada ao longo do desenvolvimento do projeto é o facto dos autores estarem a desenhar a arquitetura dos componentes ao invés de seguir uma apresentada numa unidade curricular.

\medskip

Contudo, os autores esperam que, dentro do período para a entrega do projeto, consigam concluir todos os requisitos funcionais e não funcionais do mesmo, produzindo uma plataforma que, com poucas alterações, poderia ser utilizada num contexto real e não apenas no contexto da elaboração de um trabalho final de curso.


\medskip


\subsection{Trabalho futuro}

No estado atual do mesmo, ainda faltam concluir algumas funcionalidades do projeto, como o desenvolvimento da aplicação \textit{web} e uma revisão geral sobre a aplicação móvel e a API de maneira a refinar o mesmo e prepará-lo para a entrega final.